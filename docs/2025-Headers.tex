% --- FONT AND ENCODING ---
\usepackage[T1]{fontenc}      % Specifies the output font encoding for better character rendering.
\usepackage{csquotes}         % Provides commands for advanced quotation handling.
\usepackage{fontspec}         % Allows the use of system-installed fonts (requires XeLaTeX or LuaLaTeX).
\setmainfont{Cousine}         % Sets the main document font.
\usepackage{mathtools, esint, array, nicefrac, cancel, booktabs} % Load math tools before unicode-math.
\usepackage{unicode-math}     % Provides support for Unicode math characters and fonts.
\setmathfont{Noto Sans Math}  % Sets the mathematical font.

% --- GRAPHICS AND LAYOUT ---
\usepackage{graphicx}         % For including images.
\graphicspath{{assets/Logos/}} % Specifies the directory for images.
\usepackage{latexsym, amsmath, amssymb, amsfonts, mathrsfs, amsthm, bm} % Core AMS math packages for symbols and environments.
\usepackage{multirow, longtable, tabulary, supertabular} % Advanced table formatting.
\usepackage{colortbl, multicol, comment} % For colored tables, multiple columns, and commenting out sections.
\usepackage{ragged2e}         % Provides commands for ragged text alignment.
\usepackage{tkz-berge}        % For drawing graphs from graph theory.
\usepackage[imagemagick]{sagetex}
\usepackage{Sweave}           % For embedding R code (though knitr is more common now).
\usepackage[normalem]{ulem}   % For underlining, striking out, etc.
\usepackage{wrapfig}          % Allows text to wrap around figures.
\usepackage{cuted}            % For full-width floats in a two-column document.
\usepackage{etoolbox}         % Provides tools for programming in LaTeX.
\AfterEndEnvironment{strip}{\leavevmode}
\usepackage{lipsum}           % Generates dummy text.
\usepackage{pifont, graphics} % Provides access to Zapf Dingbats and other symbols.
\usepackage{pgffor}           % Provides loop commands.
\usepackage[keepaspectratio,controls=none]{animate} % For creating animated content in PDFs.
\usepackage{caption}          % For customizing captions of figures and tables.

% --- CUSTOM COMMANDS AND ENVIRONMENTS ---
% Redefine the square root symbol for a different appearance.
\usepackage{letltxmacro}
\makeatletter
\let\oldr@@t\r@@t
\def\r@@t#1#2{
	\setbox0=\hbox{$\\oldr@@t#1{#2\\,}$}\dimen0=\ht0
	\advance\dimen0-0.3\ht0%
	\setbox2=\hbox{\vrule height\ht0 depth -\dimen0}%
	{\box0\lower0.4pt\box2}}
\LetLtxMacro{\oldsqrt}{\sqrt}
\renewcommand*{\sqrt}[2][\ ]{\oldsqrt[#1]{#2} }
\makeatother

% To-do list environment
\usepackage{enumitem,pifont}
\newlist{todolist}{itemize}{2}
\setlist[todolist]{label=$\\square$\\hspace{4pt}}
\newcommand{\cmark}{\ding{51}}
\newcommand{\xmark}{\ding{56}}
\newcommand{\Hecho}{\rlap{$\\square$}{\raisebox{2pt}{\\Large\\hspace{2pt}\\cmark}}\\hspace{1pt}}
\newcommand{\terminated}{\rlap{$\\square$}{\raisebox{1.21pt}{\\Large\\hspace{1.44pt}\\xmark}}\\hspace{3pt}}

% Matrix environment
\newenvironment{amatrix}[1]{
  \left(\begin{array}{@{}*{#1}{c}|c@{}}
}{%
  \end{array}\right)
}

% --- DATE AND TIME ---
\usepackage[useregional,showdow,calc]{datetime2}
\usepackage{datetime2-calc}
\DTMlangsetup[spanish,mexico]{} % Removed monthyearsep option

% --- HEADER AND FOOTER ---
\usepackage{fancyhdr}
\fancypagestyle{vladoddstyle}{
  \setlength{\headheight}{14.89998pt}
  \pagestyle{fancy}
  \fancyhf{}
  \fancyhead[RE]{\textcolor{blue}{\thepage}}
  \fancyhead[CE]{\textcolor{blue}{\texttt\today.}}
  \fancyhead[LE]{\textcolor{blue}{\small \StudentName}}
  \fancyfoot[RE]{\SubjectFancy}
  \fancyfoot[LE]{\textcolor{blue}{Cuenta: \StudentId}}
  \fancyhead[LO]{\textcolor{blue}{\thepage}}
  \fancyhead[CO]{\textcolor{blue}{\texttt\today.}}
  \fancyhead[RO]{\textcolor{blue}{\small \StudentName}}
  \fancyfoot[LO]{\SubjectFancy}
  \fancyfoot[RO]{\textcolor{blue}{Cuenta: \StudentId}}
  \renewcommand{\headrulewidth}{01pt}
  \renewcommand{\footrulewidth}{01pt}
  \def\headrule{\color{blue} \hrule}
  \def\footrule{\color{blue} \hrule}
}
\fancypagestyle{vladevenstyle}{
  \setlength{\headheight}{14.89998pt}
  \pagestyle{fancy}
  \fancyhf{}
  \fancyhead[LE]{\textcolor{blue}{\thepage}}
  \fancyhead[CE]{\textcolor{blue}{\texttt\today.}}
  \fancyhead[RE]{\textcolor{blue}{\small \StudentName}}
  \fancyfoot[LE]{\SubjectFancy}
  \fancyfoot[RE]{\textcolor{blue}{Cuenta: \StudentId}}
  \fancyhead[RO]{\textcolor{blue}{\thepage}}
  \fancyhead[CO]{\textcolor{blue}{\texttt\today.}}
  \fancyhead[LO]{\textcolor{blue}{\small \StudentName}}
  \fancyfoot[RO]{\SubjectFancy}
  \fancyfoot[LO]{\textcolor{blue}{Cuenta: \StudentId}}
  \renewcommand{\headrulewidth}{01pt}
  \renewcommand{\footrulewidth}{01pt}
  \def\headrule{\color{blue} \hrule}
  \def\footrule{\color{blue} \hrule}
}

\usepackage[pages=some,	scale=1.34]{background}
\backgroundsetup{
	opacity=0.34,
	angle=0,
	contents={\hspace{0pt}
		\sageplot{q}
	}
}

% --- DOCUMENT LAYOUT ---
\parindent = 13pt
\parskip = 08pt
\setlength{\columnsep}{41pt}
\setlength{\columnseprule}{0.68pt}
\newcommand{\latexcolumnseprulecolor}{\color{blue}}
\everymath{\displaystyle}
\linespread{1.1213}
\color{darkgray}

% --- TITLE PAGE ---
\title{\vspace{-21pt} \hspace*{08pt}
				\begin{minipage}[t][00pt][c]{4.1cm}\hspace*{-15pt}
					\includegraphics[width=\linewidth]{Escudo-UNAM-Azul}
				\end{minipage}\hfill
				\begin{minipage}[t][00pt][t]{11cm}
					\centering \vspace{-34pt}
					{\large \textbf{Universidad Nacional Autónoma de México.}}\\ \vspace{05pt}
					{\normalsize \textbf{Facultad de Economía.}\\ \vspace{05pt}
								División de estudios profesionales.\\ \vspace{-04pt}
								Licenciatura en Economía.}
				\end{minipage}\hfill
				\begin{minipage}[c][01.7cm][c]{4.1cm}
					\includegraphics[width=\linewidth]{Economia_escudo-Azul}
				\end{minipage}
				
$\-$ \vspace{08cm}	\\
				{\large \textbf{``\AssignmentTitle''}} 	\\
				\normalsize \ActivityType\vspace{8cm} \\ \hspace*{08pt}
				
				\begin{minipage}[c][03.4cm][t]{07.5cm}
					{\textbf{Periodo: \Period}}\\
					{\textbf{Asignatura:}}
					\begin{itemize}
						\item \SubjectName
						\begin{itemize}
							\item[\scalebox{0.41}{$\blacksquare$}] Grupo: \GroupId
							\item[\scalebox{0.41}{$\blacksquare$}] Salón: \Classroom\vspace{00pt}
							\end{itemize}
					\end{itemize}
					{\textbf{Profesor:}}
					\begin{itemize}
						\item Prof. \ProfName
						\item Adj. \AssistantName
					\end{itemize}
				\end{minipage}
					\hspace{03mm}
				\begin{minipage}[c][03.4cm][t]{10.0cm}
				  $\-$\vspace{08pt}\\
				  {\textbf{Estudiante:}}
				\begin{itemize}
				  \item \StudentName
				    \begin{itemize}
							\item[\scalebox{0.41}{$\blacksquare$}] Cuenta: \StudentId
							\item[\scalebox{0.41}{$\blacksquare$}] Correo: \href{mailto:\StudentEmailOne}{\StudentEmailOne}
							\item[\scalebox{0.41}{$\blacksquare$}] Correo: \href{mailto:\StudentEmailTwo}{\StudentEmailTwo} \vspace{-03pt}
							\end{itemize}
						\vspace{05pt}
				\end{itemize}
				{\textbf{Fecha de Entrega:}}
				\begin{itemize}
							\item \DueDate\vspace{17 pt}
				\end{itemize}
			\end{minipage}
			\vspace{-13 pt}
			
			\date{\texttt{ \GPSlocation a las \DTMcurrenttime h UTC \DTMcurrentzone\space a \today.}}
}
